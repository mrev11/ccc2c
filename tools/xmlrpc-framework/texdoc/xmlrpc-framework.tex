
\pagetitle{XMLRPC framework}{Dr. Vermes Mátyás}{2007. november}

\section{Előzmények}

E szoftver régebbi változata évek óta élesben működik különféle 
home-banking rendszerek részeként. A jelen csomag annyiban tér el
az éles változattól, hogy
az alkalmazásfüggetlen rész külön lett véve a számlavezetéstől,
ez a leválasztott rész van itt közreadva {\tt xmlrpc-framework} néven.
Emellett különféle korszerűsítések történtek:

\begin{itemize}
 \item Belekerült a CCC fejlettebb hibakezelése.
 \item Az osztálydefiníciók a class szintaktikára vannak átírva.
 \item A socket descriptorok helyett socket objektumokat használ.
 \item SSL támogatás került a rendszerbe.
 \item Portolva lett CCC3-ra.
\end{itemize}
  
%A változások által okozott esetleges hibák miatt jelenleg a
%szoftvert óvatosan szabad csak használni.


\section{Ez miez?}

XMLRPC szervereket írni nem túl nehéz. 
Ha belejön az ember, hamar abban a helyzetben találja magát,
hogy sok XMLRPC szervere van, amik  igénybe akarják venni 
egymás szolgáltatásait. Kiderül, hogy nem könnyű menedzselni
egy ilyen elosztott rendszert, ui.\ minden komponensnek ismernie
kell a rendszerben levő többi komponens elérhetőségéhez
szükséges paramétereket. 

E problémára ad viszonylag egyszerű megoldást az {\tt rpcwrapper}
nevű program. Ez egy forgalomirányító, ami a csillag topológia
központjában van, és közvetíti az XMLRPC lekérdezéseket és válaszokat
a kliensek és szerverek között, beleértve azt az esetet is, amikor
az egyik szerver kliense a másiknak. A forgalomirányító révén
a komponensek könnyen paraméterezhetők, mert mindenkinek csak
a központtal (vagyis a forgalomirányító wrapperral) kell 
közvetlen kapcsolatban lennie.

A rendszer része az {\tt rpcmonitor} kliens program, 
amivel figyelni lehet, hogy milyen szerverek/kliensek
vannak a rendszerben, és le lehet vele állítani egyes szervereket,
illetve a wrapper leállításával az egész rendszert.

A rendszer része az {\tt rpcsession} szerver, 
ami a kliensek hitelesítését végzi. Ez tipikusan egy olyan
szerver, aminek a szolgáltatását a többi szerver is igénybe
fogja venni.


\section{Session állapot kezelés}

A HTTP-re épülő technikáknál mindig téma a session állapot kezelés.
Pl. egy elemi tranzakció állhat abból, hogy a vásárló kiválaszt egy árucikket,
és azt berakja a kosarába. Ilyen elemi tranzakciók sorozatából
áll össze egy komplett vásárlás. Ámde  a HTTP protokoll olyan, 
hogy minden alkalommal felépíti majd megszakítja a hálózati kapcsolatot.
A rendszernek fel kell ismernie, ha az új hálózati kapcsolaton
egy korábbi kliens jelentkezik, és tudnia kell, hogy mit gyűjtött
össze korábban a kosarába.

Az XMLRPC framework tartalmaz egy egyszerű session állapot kezelést.
A session adatait adatbázisban tároljuk, az adatokat  kulcs
(sid) alapján keressük elő. A kliensnél csak az azonosításhoz
szükséges kulcs van. Az alkalmazás(fejlesztő) feladata 
összeállítani az adatoknak azt a körét, ami alkalmas a session
állapot teljes megőrzésére és visszaállítására.


\section{Párhuzamos kiszolgálás}

A szolgáltatásokat úgy kell megszervezni, 
hogy lehetőség legyen több kliens egyidejű kiszolgálására.
Tegyük fel pl., hogy van egy szerverünk, ami számlatörténet adatokat
ad a kliensnek. Ha az adatokat csak lassan lehet előkaparni
az archívumból, és amiatt az összes többi kliensnek várni kell,
az nem volna elfogadható.

A párhuzamos kiszolgálást úgy oldja meg az  XMLRPC framework,
hogy a várhatóan népszerű szolgáltatásokhoz (előre) több szerverpéldányt 
is elindít. A forgalomirányító ({\tt rpcwrapper}) fog választani az azonos
szolgáltatást nyújtó szerverek példányai közül. Ha talál olyan
szervert, aminek éppen nincs egy kliense sem, akkor annak továbbítja
a lekérdezést. Ha éppen minden szerver foglalt, akkor a lekérdezést
beleteszi annak a szervernek a sorába, amelyiknek a sora a legrövidebb.

Rendben, de mi van, ha egy kliens összetett tranzakciója, 
hol az egyik, hol a másik szerverhez kerül?
Erre is a session állapot kezelés ad választ. A session állapotot
leíró adatokat úgy kell meghatározni, illetve a szervereket úgy kell
megírni, hogy ne okozzon gondot, ha az összetett tranzakció menet
közben átkerül egyik szerverről másikra.
Az {\tt rpcsession} szerver például megfelel az előbbi feltételnek.


\section{Fordítás, a demók kipróbálása}

A {\tt clean.b} script minden binárist és logfilét letöröl.

A {\tt mkall.b} script mindent lefordít.


Az {\tt rpcstart.sh} script elindítja a demót. 
Elindul a wrapper, kettő darab session szerver és egy teszt szerver.

A {\tt monitor.sh} script elindítja az {\tt rpcmonitor} kliens programot,
amivel nézegetni lehet, hogy mik futnak.     
Az első sorban látszó {\tt system} a wrappert jelenti.
Ha ezt leállítjuk, akkor az összes többi szerver is kilép 
(így vannak megírva), azaz az egész rendszer leáll.
    
{\em Figyelmeztetés}: 
    A legtöbb hiba abból adódik, hogy a rendszer különböző 
    komponensei nem egyszerre indulnak és lépnek ki, ezért: 
\begin{itemize}
\item Indítás előtt ellenőrizni kell,
       hogy nincs-e korábbról beragadt szerverprocessz,
       az ilyeneket ki kell lőni. 
\item Indítás után ellenőrizni kell, hogy minden szerver elindult-e.
\item Leállítás után ellenőrizni kell,  hogy minden szerver leállt-e.
\end{itemize}

A teszteléshez átfáradunk a {\tt client} directoryba, és próbálgatjuk 
az ott található programokat.



\section{A dokumentációkról}

XMLRPC specifikációnak itt volna ez a régi írás:
\href{xmlrpc-spec.htm}{XML-RPC Specification},
ami elismerésre méltóan tömör, csak sajnos elavult.
A fő baj, hogy ezt még azzal a feltételezéssel írták, 
hogy egy XML dokumentum byte-okból áll. Ezesetben
az XMLRPC string típus valóban tartalmazhatna tetszőleges
bináris adatot. Később azonban az okosok kitalálták,
hogy az XML {\em karakterekből\/} áll, amivel a lehetőségek
valamelyest csökkentek. Az alkalmazásfejlesztőket ez annyiban
érinti, hogy a bináris adatokat nem lehet nyersen átküldeni
XMLRPC lekérdezésben vagy válaszban, hanem base-64 kódba
kell csomagolni.


Változtatás nélkül iderakom a rendszer első, 
\href{xmlrpckonto.html}{2001-es dokumentációját},
ami néhány részletében elavult ugyan, 
de a lényeget illetően még mindig informatív.



    